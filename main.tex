% Created 2017-11-21 Tue 22:36
% Intended LaTeX compiler: pdflatex
\documentclass[french]{article}
\usepackage[utf8]{inputenc}
\usepackage[T1]{fontenc}
\usepackage{graphicx}
\usepackage{grffile}
\usepackage{longtable}
\usepackage{wrapfig}
\usepackage{rotating}
\usepackage[normalem]{ulem}
\usepackage{amsmath}
\usepackage{textcomp}
\usepackage{amssymb}
\usepackage{capt-of}
\usepackage{hyperref}
\usepackage{babel}
\usepackage{minted}
\date{}
\title{\emph{Integrating Deep Learning in Expert Systems\\ for Medical Diagnosis}}
\hypersetup{
 pdfauthor={Willian Ver Valen Paiva},
 pdftitle={\emph{Integrating Deep Learning in Expert Systems\\ for Medical Diagnosis}},
 pdfkeywords={},
 pdfsubject={},
 pdfcreator={Emacs 27.0.50 (Org mode 9.1.2)}, 
 pdflang={Frenchb}}
\begin{document}

\maketitle

\section{Context}
\label{sec:orga8bb14f}

Le début du XXIème siècle a vu une augmentation spectaculaire d’applications
informatiques apportant une assistance à de nombreuses activités humaines, y
compris pour le secteur de la médecine. L’utilisation de la technologie
appliquée à la santé est devenue un des points d’intérêt de notre société, par
exemple pour l’emploi des robots d’assistances en chirurgie, la mesure des
données physiologiques comme le taux de sucre dans le sang pour les diabétiques,
ou le dossier médical partagé.  

L’objectif principal de ce projet de recherche  s’appuie sur la start up Lucine,
créée pour apporter des solutions innovantes pour les personnes qui, comme sa
fondatrice, souffrent de douleur chronique, ceci afin d’en diminuer l’impact sur
la vie quotidienne et d’en assurer une prise en charge optimisée.  L’idée
consiste à réaliser une application destinée à un usage personnalisé, permettant
de pouvoir mesurer automatiquement les niveaux de douleur ressentis par les
détenteurs de l’application envisagée. Cette solution pourra aider les médecins
à fournir un traitement adapté aux patients concernés.  

La mesure de la douleur n’est pas une tâche facile et constitue depuis longtemps
un défi, même pour des professionnels de santé avertis. De nombreuses échelles
et méthodes de mesure ont été créées pour répondre à une telle question
\cite{wong1996wong,mccaffery1999pain,portenoy1996visual,melzack1975mcgill,galer1997development,gracely1988descriptor}
mais leur efficacité reste limitée. Jusqu’à ce jour, il est encore
difficile d’arriver à une mesure précise de la douleur.  

La raison principale pour cela est que la douleur est une sensation subjective,
et la mesure la plus utilisée est l’auto-évaluation par le patient. Même si
cette mesure donne des résultats relativement acceptables, elle est encore
insuffisante sur certains aspects, notamment quand on en vient à la notion de
douleur simulée \cite{gwen2007faces}, ou, plus  couramment, lorsque le patient a
des difficultés à communiquer, comme  lorsqu’il s’agit d’enfants ou de personnes
âgées \cite{lucey2011automatically}.

La création  d’une méthode capable  de fournir une  mesure précise du  niveau de
douleur ressentie par le patient pourrait  aider le système actuel de diagnostic
à dépasser  cette barrière  difficile qu'est  la subjectivité  de la  douleur du
patient.

\subsection{État de l’art}
\label{sec:org8119c98}
Plusieurs travaux  ont déjà essayé d’obtenir  un système de détection  de la
douleur par  l’analyse des  données biométriques, telles  ques les  pupilles, la
voix, l'ECG etc.). La reconnaissance faciale est l'un des systèmes qui se révèle
le  plus   prometteur  pour  mesurer   et  analyser  la  douleur   d'un  patient
efficacement.  Afin d’atteindre cet objectif, plusieurs systèmes ont utilisé des
collections  de données  affectées  de codes  FACS (\emph{Facial Action  Coding System}) \cite{lucey2011painful}.

Le  principal  travail  réalisé sur  ce  sujet  a  été  la recherche  menée  par
l’Université de Pittsburgh "\emph{Automatically  Detecting Pain in Video Through Facial  Action Units}"  \cite{lucey2011automatically}.
Cette  recherche s’est basée  principalement sur  l’utilisation des  FACS  pour
détecter  la douleur  à partir d'images. Les résultats obtenus sont restés
limités, puisqu'ils ont porté sur la présence ou l’absence de douleur (« pain or
no pain »). 

Une autre recherche  menée sur le même  sujet avec les mêmes  données, est celle
qui a été menée à l’Université d’Aalborg, Danemark \cite{bellantonio2016spatio}.
 Cette  recherche, plus récente,  a obtenu des  résultats  plus  prometteurs,
en  utilisant  une  combinaison  de  réseaux neuronaux convolutionnels, et de
réseaux neuronaux récurrents.  En combinant ces deux méthodes, il  a été
possible de  considérer non seulement une  image et ses FACS, mais aussi
d’analyser les images  préalables de la séquence considérée, et d’obtenir  une
meilleure  précision  des  résultats. 

Cela  permettant  ainsi d’atteindre un échelonnement allant de « absence de
douleur « (« no pain ») à « forte douleur » ( « strong pain ») en  passant par «
faible douleur » (« weak pain »), ce qui a constitué une première amélioration.


Le point commun de ces recherches est le fait qu’elles utilisent la même base de
données pour l’apprentissage et cela ne permet pas de fournir une précision
suffisante pour constituer un outil vraiment fiable en adéquation des
obligations légales cliniques. De plus, ces recherches se fondent uniquement sur
les expressions faciales, et ne prennent pas en considération d’autres facteurs
comme le ton de la voix, et la sémantique utilisée dans le dialogue. Ces
facteurs permettent pourtant une analyse de la composante émotionnelle, un facteur
important lorsqu’il s’agit de mesurer convenablement le niveau de douleur du
patient \cite{hale1997emotional}.   

\subsection{Approche}
\label{sec:orgcad02fb}

Face aux limitations présentées par les recherches précédemment menées, cette
thèse propose une approche originale du problème, avec notamment l’utilisation
d’une nouvelle base de données, générée grâce à la contribution de volontaires
souffrant de douleurs chroniques (tout en respectant la réglementation CNIL
ainsi que celles concernant les données de santé) et qui servira à une
application utilisant le \emph{Deep Learning}.  

Lucine a bien perçu la nécessité, en première instance, de générer une base de
données originale et adéquate, et prépare actuellement sa constitution. Il
s’agira de collecter un certain nombre de vidéos illustrant des situations où
des personnes sont en souffrance, afin de pouvoir constituer un ensemble de
données significatives, pouvant permettre d’extraire des éléments sémantiques
pertinents d’expression de la douleur. Ainsi, les données disponibles pourront
servir de matériau pour une application de \emph{Deep Learning}, qui permettra de bien
connaître les paramètres d’une personne en situation de douleur chronique, de
diagnostiquer avec précision le niveau de douleur ressentie, et d’en optimiser
la prise en charge.

Cependant, même si une nouvelle base de données constitue un bon point de
départ, l’apprentissage d’un modèle de \emph{Deep Learning}, pour être efficace,
devrait reposer sur une collecte de données quantitativement très importante. Et
comme il ne sera matériellement pas possible de récolter autant de données que
nous en aurions besoin, nous nous proposons d’utiliser une méthode
complémentaire qui devrait permettre de circonvenir les contraintes
quantitatives de la base de données envisagée. L’idée consiste à compenser
l’insuffisance en volume de l’enquête en utilisant l’expertise des acteurs de
terrain comme les médecins et les spécialistes de la douleur, car ils ont une
longue expérience et une bonne connaissance du problème, et ne devraient pas
être exclus de l’approche proposée.  

En effet, un praticien utilise autre chose qu’une simple image pour mesurer la
douleur. Il prend en considération de nombreux autres facteurs. Le projet
propose donc d’utiliser cette expérience pour compenser le faible volume de
données d’entraînement, ce qui pourrait se comparer au fonctionnement des
systèmes experts \cite{giarratano1998expert}.  

Afin de réaliser notre objectif , nous allons donc développer des méthodes
combinant les mécanismes du \emph{Deep Learning} avec les techniques des systèmes
experts. Nous pensons définir un ensemble de règles de type "if-then-else" sur
la base de la façon dont fonctionnent les experts humains, et entraîner des
\emph{Deep Learning} à prendre des décisions à chaque nœud de l’arbre de décisions.


\section{Objectifs de la thèse}
\label{sec:orgf2f9c8d}

Comme les données sur ce sujet sont plutôt limitées, l’un des principaux
objectifs de cette thèse sera d’entraîner des systèmes de \emph{Deep Learning} à être
capables de mesurer de la façon la plus précise possible le niveau de douleur
à partir d'une vidéo, en utilisant des données réduites grâce à un cœur de facteurs de
décision importé, élaborés par des experts qui ont déjà une connaissance de la
mesure de la douleur.  

Il s’agira ainsi de rendre le processus d’apprentissage moins dépendant de la
grande quantité de données à analyser, même si on prendra garde de ne pas
sacrifier les capacités de généralisation du modèle.  


\subsection{Défis scientifiques et technologiques}
\label{sec:orgb1aac1c}


Les défis scientifiques que nous explorerons durant cette thèse sont:
\begin{itemize}
\item proposer un modèle capable de fournir efficacement une mesure de la douleur.
\item identifier les principaux points de décision qui peuvent être importés sur la
base des connaissances des experts, et qui peuvent être utilisés pour
améliorer les modèles d’apprentissage.
\end{itemize}

\section{Organisation}
\label{sec:org66b1492}
Cette thèse se déroulera sur une période de 36 mois et pendant toute cette
durée, le temps de travail se répartira entre le laboratoire hôte (le LABRI,
Université de Bordeaux), et la compagnie (Lucine) Planning
\subsection{Planning}
\label{sec:org2dc06f3}
\begin{itemize}
\item \textbf{M1 - M6 :} Discussion avec les experts humains pour créer un ensemble de règles et un système expert. Identification des conditions pour ces règles.
\item \textbf{M7 - M12 :} Collecte des données permettant d’entrainer et d’évaluer les Deep Networks testant les conditions des règles du système expert.
\item \textbf{M13 - M18 :} Entrainements et évaluations des Deep Networks à partir des données col- lectées. Premières publications (dans les domaines de l’Intelligence Articielle et médical).
\item \textbf{M19 - M24 :} Seconde itération. Nouvelles captures de données selon l’experience acquise lors de la première itération.
\item \textbf{M25 - M30 :} Nouveaux entrainements et évaluations des Deep Networks. Nouvelles pub- lications.
\item \textbf{M31 - M36 :} Rédaction et défense de la thèse.
\end{itemize}


\bibliographystyle{unsrt}
\bibliography{repport}
\end{document}