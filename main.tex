% Created 2017-10-20 Fri 14:04
% Intended LaTeX compiler: pdflatex
\documentclass[11pt]{article}
\usepackage[utf8]{inputenc}
\usepackage[T1]{fontenc}
\usepackage{graphicx}
\usepackage{grffile}
\usepackage{longtable}
\usepackage{wrapfig}
\usepackage{rotating}
\usepackage[normalem]{ulem}
\usepackage{amsmath}
\usepackage{textcomp}
\usepackage{amssymb}
\usepackage{capt-of}
\usepackage{hyperref}
\usepackage{minted}
\date{}
\title{to be defined}
\hypersetup{
 pdfauthor={Willian Ver Valen Paiva},
 pdftitle={to be defined},
 pdfkeywords={},
 pdfsubject={},
 pdfcreator={Emacs 27.0.50 (Org mode 9.1.1)}, 
 pdflang={English}}
\begin{document}

\maketitle

\usepackage{xcolor}
\definecolor{ForestGreen}{rgb}{0.0, 0.5, 0.0}
\newcommand{\vincent}[1]{{\color{ForestGreen}#1}}
\newcommand{\vincentrmk}[1]{{\color{ForestGreen}\bf #1}}


\section{Context}
\label{sec:orgf816e45}

Le début  du XXIème  siècle a vu  une augmentation  spectaculaire d’applications
informatiques apportant  une assistance  à de  nombreuses activités  humaines, y
compris  pour  le  secteur  de  la médecine.  L’utilisation  de  la  technologie
appliquée à la santé  est devenue un des points d’intérêt  de notre société, par
exemple  pour l’emploi  des robots  d'assistances  en chirurgie,  la mesure  des
données physiologiques comme le taux de sucre dans le sang pour les diabétiques,
ou le dossier médical partagé.

%% \begin{itemize}
%% \item l’emploi des robots d'assistances en chirurgie~;
%% \item la mesure des données physiologiques comme le taux de sucre dans le sang pour les diabétiques~;
%% \item les \emph{serious games} pour réapprendre à marcher~;
%% \item le dossier médical partagé.
%% \end{itemize}

%% Ces innovations qui parsèment le champ de santé émergent dans un domaine
%% important qui est celui de la gestion de la douleur. 

L’objectif  principal de  ce  projet  de recherche  consiste  à pouvoir  mesurer
automatiquement  les  niveaux  de  douleur.  Cette
solution   pourra  aider  les médecins  à fournir  un  traitement adapté  aux
patients affectés de douleurs chroniques. La  mesure de la douleur n’est pas une
tâche facile et constitue depuis longtemps un défi, même pour des professionnels
de santé  avertis. De nombreuses échelles  et méthodes de mesure  ont été créées
pour             répondre à            une             telle             question
\cite{wong1996wong,mccaffery1999pain,portenoy1996visual,melzack1975mcgill,galer1997development,gracely1988descriptor}
mais leur  efficacité reste limitée.  Jusqu’à  ce jour, il est  encore difficile
d’arriver à une mesure précise de la douleur.

La raison principale pour cela est  que la douleur est une sensation subjective,
et la  mesure la plus  utilisée est l’auto-évaluation  par le patient.   Même si
cette  mesure donne  des  résultats relativement  acceptables,  elle est  encore
insuffisante sur  certains aspects, notamment quand  on en vient à  la notion de
douleur simulée~\cite{gwen2007faces}, ou, plus  couramment, lorsque le patient a
des difficultés à communiquer, comme  lorsqu’il s’agit d’enfants ou de personnes
âgées~\cite{lucey2011automatically}.

La création  d’une méthode capable  de fournir une  mesure précise du  niveau de
douleur ressentie par le patient pourrait  aider le système actuel de diagnostic
à dépasser  cette barrière  difficile qu'est  la subjectivité  de la  douleur du
patient.


\subsection{\'Etat de l'art}
\label{sec:org08f9dfb}

Plusieurs travaux  ont d\'ej\`a essayé d’obtenir  un système de détection  de la
douleur par  l’analyse des  données biométriques, telles  ques les  pupilles, la
voix, l'ECG etc.). La reconnaissance faciale est l'un des systèmes qui se révèle
le  plus   prometteur  pour  mesurer   et  analyser  la  douleur   d'un  patient
efficacement.  Afin d’atteindre cet objectif, plusieurs systèmes ont utilisé des
collections  de données  affectées  de codes  FACS~(\emph{Facial Action  Coding
  System})~\cite{lucey2011painful}.

Le  principal  travail  réalisé sur  ce  sujet  a  été  la recherche  menée  par
l’Université de Pittsburgh "\emph{Automatically  Detecting Pain in Video Through
  Facial  Action Units}"  \cite{lucey2011automatically}.  Cette  recherche s’est
basée  principalement sur  l’utilisation des  FACS  pour détecter  la douleur  à
partir d'images. Les résultats obtenus sont restés limités, puisqu'ils ont porté
sur la présence ou l’absence de douleur (« pain or no pain »).

Une autre recherche  menée sur le même  sujet avec les mêmes  données, est celle
qui        a       été        menée        à       l’Université        d’Aalborg
Danemark~\cite{bellantonio2016spatio}. Cette  recherche, plus récente,  a obtenu
des  résultats  plus  prometteurs,  en  utilisant  une  combinaison  de  réseaux
neuronaux convolutionnels, et de réseaux neuronaux récurrents.  En combinant ces
deux méthodes, il  a été possible de  considérer non seulement une  image et ses
FACS, mais aussi d’analyser les images  préalables de la séquence considérée, et
d’obtenir  une   meilleure  précision  des  résultats.   Cela  permettant  ainsi
d’atteindre un échelonnement allant de « absence de  douleur « (« no pain ») à «
forte douleur » ( « strong pain ») en  passant par « faible douleur » (« weak pain
»), ce qui a constitué une première amélioration.

Le point commun de ces recherches est le fait qu’elles utilisent la même base de
données pour l’apprentissage et cela ne permet pas de fournir une précision
suffisante pour constituer un outil vraiment fiable en adéquation des
obligations légales cliniques. De plus, ces recherches se fondent uniquement sur
les expressions faciales, et ne prennent pas en considération d’autres facteurs
comme le ton de la voix, et la sémantique utilisée dans le dialogue. Ces
facteurs permettent pourtant une analyse de la composante émotionnelle, un facteur
important lorsqu’il s’agit de mesurer convenablement le niveau de douleur du
patient~\cite{hale1997emotional}.   


\subsection{Approche}
\label{sec:org97cce26}

\vincentrmk{L'approche devrait etre mieux précisée : 1) pour compenser
  l'absence de  donnees, Lucine  a créée/crée une  base de  donnees et
  details sur  la base  de donnees;  2) on  souhaite utiliser  le deep
  learning pour predire la douleur; 3) le deep learning a sans doute
  besoin de plus de donnees qu’on peut en creer; 4) l'idee de depart
  de la these est d'integrer la connaissance des experts humains sous
  forme de systeme expert; 5) nous allons donc developper des methodes
pour combiner deep learning et systeme expert.}

Cette thèse propose une nouvelle approche du problème avec l’utilisation d’une
nouvelle base de données générée grâce à la contribution de volontaires
souffrant de douleurs chroniques (réglementation CNIL / données de santé
respectées).

Mais si une nouvelle base de données constitue un bon point de
départ, l’apprentissage d’un modèle de \emph{Deep Learning} peut exiger d’importantes
quantités de données. C’est pourquoi nous proposons d'intégrer l'expérience
des praticiens. 

En effet, un praticien n'utilise pas que son sens de la vue pour mesurer la
douleur. Il prend en considération de nombreux autres facteurs. C’est pourquoi
le projet propose d’utiliser l’expérience des praticiens pour compenser le
faible volume de données d’entraînement, ce qui pourrait se comparer au
fonctionnement des systèmes experts \cite{giarratano1998expert}. 

Cependant, il n’est pas encore clair à ce jour quelle sera la manière dont il
sera possible de concilier le \emph{Deep Learning} et les systèmes experts. Le but de
cette thèse sera l’exploration des moyens d’y parvenir. 

\section{Objectifs de la thèse}
\label{sec:orgf1f9281}

Comme les données sur ce sujet sont plutôt limitées, l’un des principaux
objectifs de cette thèse sera d’entraîner des systèmes de \emph{Deep Learning} à être
capables de mesurer de la façon la plus précise possible le niveau de douleur
à partir d'une vidéo, en utilisant des données réduites grâce à un c\oe{}ur de facteurs de
décision importé, élaborés par des experts qui ont déjà une connaissance de la
mesure de la douleur.  

Il s’agira ainsi de rendre le processus d’apprentissage moins dépendant de la
grande quantité de données à analyser, même si on prendra garde de ne pas
sacrifier les capacités de généralisation du modèle.  


\subsection{Défis scientifiques et technologiques}
\label{sec:org70ad40f}

Les défis scientifiques que nous explorerons durant cette thèse sont~:
\begin{itemize}
\item proposer un modèle capable de fournir efficacement une mesure de la douleur
\item identifier les principaux points de décision qui peuvent être importés sur
la base des connaissances des experts, et qui peuvent être utilisés pour
améliorer les modèles d’apprentissage.
\end{itemize}


\section{Organization}
\label{sec:org1cc9eab}
Cette thèse se déroulera sur une période de 36 mois et pendant toute cette
durée, le temps de travail se répartira entre le laboratoire hôte (le LABRI,
Université de Bordeaux), et la compagnie (Lucine).

\subsection{Planning}
\label{sec:org8a30be3}
\begin{itemize}
\item T0 + 6m:
\item T0 + 12m:
\item T0 + 18m:
\item T0 + 24m:
\item T0 + 36m: Writing of thesis and defense
\end{itemize}



\bibliographystyle{unsrt}
\bibliography{repport}
\end{document}
